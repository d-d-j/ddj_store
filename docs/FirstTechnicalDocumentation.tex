 %%%%%%%%%%%%%%%%%%%%%%%%%%%%%%%%%%%%%%%%%
% Short Sectioned Assignment
% LaTeX Template
% Version 1.0 (5/5/12)
%
% This template has been downloaded from:
% http://www.LaTeXTemplates.com
%
% Original author:
% Frits Wenneker (http://www.howtotex.com)
%
% License:
% CC BY-NC-SA 3.0 (http://creativecommons.org/licenses/by-nc-sa/3.0/)
%
%%%%%%%%%%%%%%%%%%%%%%%%%%%%%%%%%%%%%%%%%

%----------------------------------------------------------------------------------------
%   PACKAGES AND OTHER DOC%%UMENT CONFIGURATIONS
%----------------------------------------------------------------------------------------

\documentclass[paper=a4, fontsize=11pt]{scrartcl} % A4 paper and 11pt font size

%\usepackage[final]{pdfpages}
\usepackage[utf8]{inputenc}
\usepackage[T1]{fontenc} % Use 8-bit encoding that has 256 glyphs
%\usepackage{fourier} % Use the Adobe Utopia font for the document - comment this line to return to the LaTeX default
\usepackage[polish]{babel} % English language/hyphenation
\usepackage{amsmath,amsfonts,amsthm} % Math packages

\usepackage[usenames,dvipsnames]{color} % Required for custom colors
\usepackage{graphicx}
\usepackage{caption}
\usepackage{subcaption}
\usepackage{listings} % Required for insertion of code

\usepackage{lastpage}
\usepackage{tabularx}

%\usepackage{todonotes}

\usepackage{sectsty} % Allows customizing section commands
\allsectionsfont{ \normalfont\scshape} % Make all sections the default font and small caps

\usepackage{fancyhdr} % Custom headers and footers
\pagestyle{fancyplain} % Makes all pages in the document conform to the custom headers and footers
\fancyhead{} % No page header - if you want one, create it in the same way as the footers below
\fancyfoot[L]{} % Empty left footer
\fancyfoot[C]{} % Empty center footer
\fancyfoot[R]{\thepage \, z \pageref{LastPage}} % Page numbering for right footer
\renewcommand{\headrulewidth}{0pt} % Remove header underlines
\renewcommand{\footrulewidth}{0pt} % Remove footer underlines
\setlength{\headheight}{6.6pt} % Customize the height of the header



\numberwithin{equation}{section} % Number equations within sections (i.e. 1.1, 1.2, 2.1, 2.2 instead of 1, 2, 3, 4)
\numberwithin{figure}{section} % Number figures within sections (i.e. 1.1, 1.2, 2.1, 2.2 instead of 1, 2, 3, 4)
\numberwithin{table}{section} % Number tables within sections (i.e. 1.1, 1.2, 2.1, 2.2 instead of 1, 2, 3, 4)

%\setlength\parindent{0pt} % Removes all indentation from paragraphs - comment this line for an assignment with lots of text

%----------------------------------------------------------------------------------------
%   CODE INCLUSION CONFIGURATION
%----------------------------------------------------------------------------------------

\definecolor{MyDarkGreen}{rgb}{0.0,0.4,0.0} % This is the color used for comments
\lstloadlanguages{C++} % Load Cpp syntax for listings, for a list of other languages supported see: ftp://ftp.tex.ac.uk/tex-archive/macros/latex/contrib/listings/listings.pdf
\lstset{language=C++, % UseCpp
        frame=single, % Single frame around code
        basicstyle=\small\ttfamily, % Use small true type font
        keywordstyle=[1]\color{Blue}\bf, % Perl functions bold and blue
        keywordstyle=[2]\color{Purple}, % Perl function arguments purple
        keywordstyle=[3]\color{Blue}\underbar, % Custom functions underlined and blue
        identifierstyle=, % Nothing special about identifiers
        commentstyle=\usefont{T1}{pcr}{m}{sl}\color{MyDarkGreen}\small, % Comments small dark green courier font
        stringstyle=\color{Purple}, % Strings are purple
        showstringspaces=false, % Don't put marks in string spaces
        tabsize=5, % 5 spaces per tab
        %
        % Put standard Perl functions not included in the default language here
        morekeywords={rand},
        %
        % Put Cppl function parameters here
        morekeywords=[2]{on, off, interp},
        %
        % Put user defined functions here
        morekeywords=[3]{test},
        %
        morecomment=[l][\color{Blue}]{...}, % Line continuation (...) like blue comment
        numbers=left, % Line numbers on left
        firstnumber=1, % Line numbers start with line 1
        numberstyle=\tiny\color{Blue}, % Line numbers are blue and small
        stepnumber=5 % Line numbers go in steps of 5
}

% add new command to include cpp code
\newcommand{\cppscript}[2]{
\begin{itemize}
\item[]\lstinputlisting[caption=#2,label=#1]{#1.cpp}
\end{itemize}

}

%----------------------------------------------------------------------------------------
%   TITLE SECTION
%----------------------------------------------------------------------------------------

\newcommand{\horrule}[1]{\rule{\linewidth}{#1}} % Create horizontal rule command with 1 argument of height

\title{
\vspace*{\fill}
\normalfont
\textsc{Projekt Zespołowy}\\ [20pt]
\horrule{1.5pt} \\[0.4cm] % Thin top horizontal rule
\LARGE Baza danych szeregów czasowych implementowana na klastrze obliczeniowym procesorów graficznych
\horrule{1.5pt} \\[0.1cm] % Thick bottom horizontal rule
\normalsize
\textsc{Wstępna Specyfikacja Techniczna} \\ [20pt]
\vspace*{\fill}
}

\author{Jakub Dutkowski \\ Karol Dzitkowski \\ Tomasz Janiszewski } % Your name

\date{\normalsize\today} % Today's date or a custom date

\begin{document}
\maketitle

\thispagestyle{empty}
\clearpage

\tableofcontents
\listoffigures

\chapter{}

\clearpage

\vspace{4em}


\section{Abstrakt}
Niniejsza dokumentacja przedstawia wstępny opis techniczny systemu tworzonego w ramach projektu zespołowego, którego efektem ma być praca inżynierska.
Dokument zawiera powierzchowny opis struktury systemu od strony technicznej wraz z diagramami przepływu danych oraz wydzielonymi modułami systemu.

\section{Opis programu}
Celem projektu jest stworzenie bazy danych szeregów czasowych efektywnie korzystającej z wydajności oferowanej przez współczesne
procesory graficzne. Dane powinny być przechowywane i przetwarzane w pamięci kart graficznych. Baza danych oprócz przechowywania
danych powinna umożliwiać szybką ich agregację oraz filtrowanie.

\section{Ogólna wizja struktury systemu}
\begin{itemize}
	\item System będzie składał się z dwóch podsystemów:
		\begin{enumerate}
			\item Systemu serwera głównego zwanego Master
			\item Systemu działającego na serwerach bazodanowych - węzłach (Nodes)
		\end{enumerate}
	\item Komunikacja pomiędzy Masterem a węzłami odbywa się za pomocą tcp/ip po sieci wewnętrznej, gdzie informacje przesyłane są za pomocą 
		zserializowanych binarnie struktur danych. Klienci komunikują się tylko z Masterem który oferuje im Restowe api. 
	\item Dla klientów utworzone są serwisy:
		\begin{enumerate}
			\item Serwis danych (insertService)
				\begin{itemize}
					\item Insert (możliwe rozszerzenie o InsertMany)
					\item Flush
				\end{itemize}
			\item Serwis zapytań (queryService) - zawiera wszystkie zapytania typu select zawarte w ''Specyfikacji Technicznej``
			\item Serwis statusu (statusService)
				\begin{itemize}
					\item GetLastError
				\end{itemize}
		\end{enumerate}
	\item Master zajmuje się odbieraniem zapytań od klientów, kolejkowaniem ich oraz zlecaniem ich wykonania przez węzły. Zarządza oraz monitoruje 
		węzły sprawdzając ich parametry, np. zajętość miejsca na karcie lub obciążenie. Nadzoruje wykonanie wszystkich zapytań oraz zbiera ich wyniki 
		wykonując na nich ostatni etap operacji Reduce (agregując wyniki). Gotowa odpowiedź na zapytanie kierowana jest z powrotem do klienta.
	\item Węzeł loguje się do Mastera jako gotowy do pracy węzeł. Zalogowany węzeł otrzymuje tyle identyfikatorów ile ma odpowiednich dla naszego systemu 
		(Cuda capability >= 2.0) kart graficznych NVIDIA. Główną rolą każdego węzła jest składowanie danych na kartach graficznych. Wykonują zlecane im 
		zadania (przez Mastera) wrzucając dane na odpowiednią kartę lub wykonując zapytanie na danych znajdujących się już na karcie. Takie dane mogą 
		być agregowane, sortowane oraz filtrowane na sposoby wymienione w "Specyfikacji Technicznej". W tym momencie kożystać będziemy z wydajności 
		kart graficznych. Węzeł komunikuje się tylko i wyłącznie z Masterem.
\end{itemize}
\begin{figure}[t]
	\begin{center}
		\caption{Ogólna struktura systemu}
 		\makebox[\textwidth]{\includegraphics[width=\paperwidth]{Deployment_Diagram}}
	\end{center}
\end{figure}
\clearpage

\section{Master}
	Program głównego serwera napisany jest w języku programowania The Go Programming Language (http://www.golang.org). 
	Program korzysta między innymi z takich pakietów jak: ``log4go'' - do logowania, ``gorest'' - wspomagający Restowe API czy ``gcfg'' - do konfiguracji.
	\subsection{Struktura}
		\begin{figure}[t]
			\begin{center}
				\caption{Struktura serwera głównego}
 				\makebox[\textwidth]{\includegraphics[width=\paperwidth]{Component_Master}}
			\end{center}
		\end{figure}
	\subsection{Przepływ danych}
		\begin{figure}[t]
			\begin{center}
				\caption{Przepływ danych dla serwera głównego}
 				\makebox[\textwidth]{\includegraphics[width=\paperwidth]{DataFlow_Master}}
			\end{center}
		\end{figure}

\section{Node}
	Program węzła napisany jest w języku programowania C++ w wersji C++11 z użyciem bibliotek BOOST i LOG4CPLUS. Do obsługi kart 
	graficznych używane są: Cuda runtime api oraz Thrust. Program pisany jest w oparciu o CUDA toolkit 5.5 wraz z najnowszymi sterownikami. 
	Program używa kart graficznych o CUDA compute capability $>= 2.0$ i aby program działał niezbędna jest przynajmniej jedna taka karta na węźle.   
	\subsection{Struktura}
		Program węzła składa się z wielu współpracujących ze sobą "modułów" którym odpowiadają klasy w programie. Na każdym węźle może znajdować się
		wiele kart graficznych, które będą używane przez program jeśli spełniają wymagania. Dane będziemy wkładać na karty oraz wykonywać na nich zapytania
		niezależnie od siebie, tj. na poziomie obsługi karty nie mamy pojęcia o istnieniu innych. Wymyślona struktura programu ma na celu zapewnienie dużej 
		równoległości wykonywania zadań oraz modułowości komponentów. \\
		Opis wybranych elementów struktury:
		\begin{itemize}
			\item Node - "Korzeń" całego programu dla węzła. Stworzenie instancji tej klasy spowoduje rozpoczęcie działania węzła. \\ \\
				Funkcjonalność: 
				\begin{itemize}
					\item Odpowiada za uruchomienie i zatrzymianie systemu, w tym tworzy instancję klasy StoreController dla każdej odpowiedniej karty
						graficznej, którą dany StoreController będzie obsługiwał.
					\item Przyjmuje zadania przychodzące od Mastera i zleca je odpowiednim StoreControllerom
					\item W ramach tej klasy działa wątek "zbierający" ukończone zadania i wysyłający ich rezultaty do Mastera
				\end{itemize}
			\item Store Controller - odpowiada za obsługę jednej karty graficznej o podanym numerze. \\ \\
				Funkcjonalność: 
				\begin{itemize}
					\item Alokuje główną tablicę na dane na karcie graficznej w której przechowywane będą dane
					\item Wykonuje zadania przydzielone mu przez Node'a
					\item Zarządza strukturami do wykonywania zadań: wieloma obiektami Store Buffer (jeden dla każdej metryki) oraz obiektem Query Monitor
				\end{itemize}
			\item Store Buffer - obsługuje wpływające rekordy z jedną metryką \\ \\
				Funkcjonalność: 
				\begin{itemize}
					\item Buforuje przychodzące dane w specjalnym buforze
					\item Zamienia bufory jeśli główny się napełnił i zleca załadowanie przechowywanych w buforze danych na GPU
					\item Składuje informacje o położeniu danych w pamięci GPU w B+ drzewie poprzez obiekt BTreeMonitor
				\end{itemize}
			\item Network Client (Node communication Client) - odpowiada za komunikację z Masterem \\ \\
				Funkcjonalność: 
				\begin{itemize}
					\item Loguje węzeł do serwera głównego
					\item Odbiera zlecenia wykonania zadań od Mastera
					\item Wysyła rezultat wykonanego zadania do Mastera
				\end{itemize}
			\item Node Controller (Resource Controller) - nadzoruje wykorzystanie zasobów na węźle i przesyła ja do Mastera \\ \\
				Funkcjonalność: 
				\begin{itemize}
					\item Pobiera dane o wykorzystaniu CPU, RAM, GPU
					\item Pobiera dane o ilości wolnego i zajętego miejsca w głównej tablicy danych na GPU
					\item Wysyła statystyki do Mastera
				\end{itemize}
			\item Inne moduły nazwane X Monitor mają za zadanie zabezpieczyć dostęp do pewnego zasobu związanego z X przed dostępem w wielu wątków. 
				Taki obiekt monitora używa najczęściej Mutexa albo Semafora oraz Condition Variable do zarządzania dostępem do ochranianego zasobu.
		\end{itemize}
		\begin{figure}[t]
			\begin{center}
				\caption{Struktura węzła}
 				\makebox[\textwidth]{\includegraphics[width=\paperwidth]{Component_Node}}
			\end{center}
		\end{figure}

	\subsection{Przepływ danych}
		Przepływ danych w węźle przy wykonywaniu operacji Insert (wkładania danych do bazy danych):
		\begin{enumerate}
			\item Node Communication Client otrzymuje TaskRequest od Mastera zawierający typ zadania (INSERT), identyfikator mówiący 
				na którą kartę dane mają być wsadzone oraz element do załadowania do bazy.
			\item TaskRequest poprzez sygnał (boost.Signals) przekazywany jest do Node'a który przekazuje TaskRequest do Task Monitora ten tworzy 
				obiekt zadania (StoreTask)
			\item StoreTask przekazywany jest po identyfikatorze do odpowiedniego StoreControllera odpowiadającego za kartę w której umieszczone mają być dane
			\item StoreController wykonuje zadanie przekazując element do załadowania do bazy do bufora (StoreBuffer)
			\item StoreController wykonuje na zadaniu SetResult z pozytywnym wynikiem więc zadanie będzie mogło być zakończone i usunięte
			\item Jeśli bufor w danym obiekcie StoreBuffer  stanie się pełny bufor zostaje zamieniony na pusty, a pełny trafia do Upload Monitora który załaduje dane na GPU
			\item Po udanym załadowaniu danych na GPU informacje o położeniu danychna karcie zwrócone przez UploadMonitor w postaci struktury infoElement przekazywane są do
				BTree Monitora w celu umieszczenia ich w B+Drzewie 
			\item Dopiero po ukończeniu tych wszystkich kroków można uznać daną za włożoną do bazy danych
		\end{enumerate}
		Przepływ danych w węźle przy wykonywaniu zapytania typu Select (pobieranie danych z bazy danych):
		\begin{enumerate}
			\item Node Communication Client otrzymuje TaskRequest od Mastera zawierający typ zadania $(SELECT_ABC)$ oraz dane do zapytania (np. filtry)
			\item TaskRequest poprzez sygnał (boost.Signals) przekazywany jest do Node'a który przekazuje TaskRequest do Task Monitora ten tworzy 
				obiekt zadania (StoreTask)
			\item StoreTask przekazywany jest do wszystkich StoreControllerów
			\item Store Controller przekazuje do swojego wątku odpowiedzialnego za zapytania zadanie do wykonania nazwijmy go queryThread. 
			\item queryThread prosi odpowiednie obiekty Store Bufferów o informacje na temat położenia danych w pamięci GPU i dostaje je pobrane z B+drzew w postaci listy infoElementów
			\item queryThread tworzy obiekt QueryRequest na podstawie zapytania i infoElementów i przekazuje go do Query Monitora, który wykonuje operację query i zwraca mu QueryResult. 
			\item Wątek wykonujący zapytania na podstawie QueryResult wywołuje operację SetResult na zadaniu które może być następnie uznane za wykonane
			\item Wątek w obiekcie Node odpowiedzialny za zarządzanie ukończonymi zadaniami orzyma takie zadanie z Task Monitora po wykonaniu na nim funkcji PopCompletedTasks
				i może pobrać resultat z danego zadania oraz wysłać go do Mastera 
		\end{enumerate}
		\begin{figure}[t]
			\begin{center}
				\caption{Przepływ danych dla węzła}
 				\makebox[\textwidth]{\includegraphics[width=\paperwidth]{DataFlow_Node}}
			\end{center}
		\end{figure}

	\subsection{System od strony GPU}
		W tym przypadku również rozpatrywać będziemy działanie systemu osobno pod względem wrzucania danych do bazy i pod względem zapytań typu Select. \\ \\
		Insert: \\
		\begin{enumerate}
			\item Upload Monitor kopiuje bufor otrzymany od StoreBuffera do bufora po stronie GPU. Dopiero po tym kroku bufor w StoreBufferze może być ponownie zamieniony.
			\item Następnie Upload Monitor wywołuje Compressor aby ten skompresował bufor znajdujący się na GPU i zwrócił mu wskaźnik na skompresowane dane.
			\item Skompresowane dane są kopiowane w odpowiednie miejsce w głównej tablicy GPU po czym informacja o położeniu tych danych zwracana jest w postaci 
				obiektu infoElement
		\end{enumerate}
		Select: \\
		\begin{enumerate}
			\item Query Monitor po otrzymaniu QueryRequest od Store Controllera wywołuje Compressor aby ten rozkompresował dane na które wskazują struktury infoElement zawarte w zapytaniu
			\item Następnie QueryMonitor wywołuje odpowiedni dla danego zapytania kernel wykonujący się na GPU przekazując mu wskaźnik do rozkompresowanych danych 
			\item Dane mogą być odpowiednio filtrowane w zależności od zapytania przez funkcję filtrującą wywołaną z kernela
			\item Wskaźnik na rezultat zapytania zwracany jest do QueryMonitora który ściąga dane z GPU i przesyła z powrotem do StoreControllera
		\end{enumerate}
		\begin{figure}[t]
			\begin{center}
				\caption{Działanie bazy od strony karty graficznej}
 				\makebox[\textwidth]{\includegraphics[width=\paperwidth]{Cuda_Store_Structure}}
			\end{center}
		\end{figure}
	\subsection{Strumienie CUDA dla operacji Insert}
		Do zarządzania strumieniami CUDA zarówno dla operacji Insert jak i Select wydzielona została klasa CudaController. Zarządzanie strumieniami zostało zaimplementowane tak aby:
		\begin{itemize}
			\item Ilość strumieni przeznaczona dla operacji Insert jak również Select była łatwo konfigurowalna. 
			\item Dla konkretnego bufora operacje kopiowania bufora na GPU oraz późniejszej kompresji załadowanego bufora wykonywały się w jednym strumieniu
			\item Dla operacji kopiowania skompresowanych buforów do głównej tabeli GPU wyróżniony był jeden strumień i wszystkie takie kopiowania były wykonywane tylko w nim.
		\end{itemize}
		\begin{figure}[t]
			\begin{center}
				\caption{Zarządzanie strumieniami CUDA przy operacji Insert}
 				\makebox[\textwidth]{\includegraphics[width=\paperwidth]{Cuda_Upload_Streams}}
			\end{center}
		\end{figure}

\section{Map-Reduce}
Zaletą systemu będzie implementacja uproszczonego stylu programowania ``Map-Reduce'', jako że jest to rozproszona baza danych.
\begin{itemize}
	\item Każdy z węzłów wyszukuje w swojej bazie odpowiednie fragmenty danych wykonuje operację Map tj. filtruje dane pod kątem pewnych 
		kryteriów, przy czym jeśli ilość dostępnej pamięci na karcie na to pozwala wykonywać się będzie tylko jedna operacja Map per węzeł dla danego zapytania.
		Dane uzyskane poprzez wykonanie operacji map są parami klucz-wartość, gdzie kluczem jest id danego zadania, a wartościami wyniki zapytania. Wyniki
		operacji Map z każdego węzła przekazywane są do Mastera w celu wykonania na nich operacji Reduce która wszystkie agreguje dane zwrócone dla konkretnego zadania. 
		\begin{figure}[t]
			\begin{center}
				\caption{Ogólny schemat Map-Reduce w systemie}
 				\makebox[\textwidth]{\includegraphics[width=0.9\textwidth, height=0.7\textwidth]{MapReduce}}
			\end{center}
		\end{figure}
	\item Każdy z węzłów poprzez użycie Mappera który korzystając z wiedzy o położeniu danych w pamięci na karcie graficznej przechowywanej w B+ drzewie filtruje dane tworzy
		pary taskId - record. W ten sposób Mapper wykonuje operację map. Dane te mogą być dodatkowo zagregowane jeszcze na karcie przez Composer'a. Wykonuje on na zmapowanych
		parach operację Compose - nazywaną przez nas operacją Reduce ponieważ jest ona analogiczna do tej którą ma wykonać na końcu Master. Tutaj mamy zamiar wykorzystać
		moc obliczeniową kart graficznych. Uzyskane dane z wielu kart graficznych mogą być wstępnie zagregowane już na węźle jeśli pozwala na to rodzaj operacji Reduce - musi być łączna i przemienna.
		\begin{figure}[t]
			\begin{center}
				\caption{Schemat działania Map-Reduce w węźle}
		 		\makebox[\textwidth]{\includegraphics[width=0.9\textwidth, height=0.7\textwidth]{MapReduce_Node}}
			\end{center}
		\end{figure}
\end{itemize}
Taki model systemu może zapewnić dużą skalowalność systemu. W przyszłości operacje Reduce Mastera będą mogły być wykonywane również na ``zarezerwawanej'' dla niego 
karcie graficznej (jeśli okaże się to korzystne).

\section{Pierwszy prototyp programu}
	\subsection{Opis funkcjonalności}
	\subsection{Przeprowadzone testy}

\section{Przyszłość}

\end{document}