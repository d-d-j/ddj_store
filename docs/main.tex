 %%%%%%%%%%%%%%%%%%%%%%%%%%%%%%%%%%%%%%%%%
% Short Sectioned Assignment
% LaTeX Template
% Version 1.0 (5/5/12)
%
% This template has been downloaded from:
% http://www.LaTeXTemplates.com
%
% Original author:
% Frits Wenneker (http://www.howtotex.com)
%
% License:
% CC BY-NC-SA 3.0 (http://creativecommons.org/licenses/by-nc-sa/3.0/)
%
%%%%%%%%%%%%%%%%%%%%%%%%%%%%%%%%%%%%%%%%%

%----------------------------------------------------------------------------------------
%   PACKAGES AND OTHER DOC%%UMENT CONFIGURATIONS
%----------------------------------------------------------------------------------------

\documentclass[paper=a4, fontsize=11pt]{scrartcl} % A4 paper and 11pt font size


\usepackage[utf8]{inputenc}
\usepackage[T1]{fontenc} % Use 8-bit encoding that has 256 glyphs
%\usepackage{fourier} % Use the Adobe Utopia font for the document - comment this line to return to the LaTeX default
\usepackage[polish]{babel} % English language/hyphenation
\usepackage{amsmath,amsfonts,amsthm} % Math packages

\usepackage[usenames,dvipsnames]{color} % Required for custom colors
\usepackage{graphicx}
\usepackage{caption}
\usepackage{subcaption}
\usepackage{listings} % Required for insertion of code

\usepackage{lastpage}
\usepackage{tabularx}

\usepackage{sectsty} % Allows customizing section commands
\allsectionsfont{ \normalfont\scshape} % Make all sections the default font and small caps

\usepackage{fancyhdr} % Custom headers and footers
\pagestyle{fancyplain} % Makes all pages in the document conform to the custom headers and footers
\fancyhead{} % No page header - if you want one, create it in the same way as the footers below
\fancyfoot[L]{} % Empty left footer
\fancyfoot[C]{} % Empty center footer
\fancyfoot[R]{\thepage \, z \pageref{LastPage}} % Page numbering for right footer
\renewcommand{\headrulewidth}{0pt} % Remove header underlines
\renewcommand{\footrulewidth}{0pt} % Remove footer underlines
\setlength{\headheight}{6.6pt} % Customize the height of the header



\numberwithin{equation}{section} % Number equations within sections (i.e. 1.1, 1.2, 2.1, 2.2 instead of 1, 2, 3, 4)
\numberwithin{figure}{section} % Number figures within sections (i.e. 1.1, 1.2, 2.1, 2.2 instead of 1, 2, 3, 4)
\numberwithin{table}{section} % Number tables within sections (i.e. 1.1, 1.2, 2.1, 2.2 instead of 1, 2, 3, 4)

%\setlength\parindent{0pt} % Removes all indentation from paragraphs - comment this line for an assignment with lots of text

%----------------------------------------------------------------------------------------
%   CODE INCLUSION CONFIGURATION
%----------------------------------------------------------------------------------------

\definecolor{MyDarkGreen}{rgb}{0.0,0.4,0.0} % This is the color used for comments
\lstloadlanguages{C++} % Load Cpp syntax for listings, for a list of other languages supported see: ftp://ftp.tex.ac.uk/tex-archive/macros/latex/contrib/listings/listings.pdf
\lstset{language=C++, % UseCpp
        frame=single, % Single frame around code
        basicstyle=\small\ttfamily, % Use small true type font
        keywordstyle=[1]\color{Blue}\bf, % Perl functions bold and blue
        keywordstyle=[2]\color{Purple}, % Perl function arguments purple
        keywordstyle=[3]\color{Blue}\underbar, % Custom functions underlined and blue
        identifierstyle=, % Nothing special about identifiers
        commentstyle=\usefont{T1}{pcr}{m}{sl}\color{MyDarkGreen}\small, % Comments small dark green courier font
        stringstyle=\color{Purple}, % Strings are purple
        showstringspaces=false, % Don't put marks in string spaces
        tabsize=5, % 5 spaces per tab
        %
        % Put standard Perl functions not included in the default language here
        morekeywords={rand},
        %
        % Put Cppl function parameters here
        morekeywords=[2]{on, off, interp},
        %
        % Put user defined functions here
        morekeywords=[3]{test},
        %
        morecomment=[l][\color{Blue}]{...}, % Line continuation (...) like blue comment
        numbers=left, % Line numbers on left
        firstnumber=1, % Line numbers start with line 1
        numberstyle=\tiny\color{Blue}, % Line numbers are blue and small
        stepnumber=5 % Line numbers go in steps of 5
}

% add new command to include cpp code
\newcommand{\cppscript}[2]{
\begin{itemize}
\item[]\lstinputlisting[caption=#2,label=#1]{#1.cpp}
\end{itemize}

}

%----------------------------------------------------------------------------------------
%   TITLE SECTION
%----------------------------------------------------------------------------------------

\newcommand{\horrule}[1]{\rule{\linewidth}{#1}} % Create horizontal rule command with 1 argument of height

\title{
\vspace*{\fill}
\normalfont
\textsc{Projekt Zespołowy}\\ [20pt]
\horrule{1.5pt} \\[0.4cm] % Thin top horizontal rule
\LARGE Baza danych szeregów czasowych implementowana na klastrze obliczeniowym procesorów graficznych
\horrule{1.5pt} \\[0.1cm] % Thick bottom horizontal rule
\normalsize
\textsc{Specyfikacja Funkcjonalna} \\ [20pt]
\vspace*{\fill}
}

\author{Jakub Dutkowski \\ Karol Dzitkowski \\ Tomasz Janiszewski } % Your name

\date{\normalsize\today} % Today's date or a custom date

\begin{document}
\maketitle

\thispagestyle{empty}
\clearpage

\tableofcontents
\listoffigures

\chapter{}

\clearpage

\vspace{4em}


\section{Abstrakt}
Niniejsza dokumentacja przedstawia wstępne założenia dotyczące projektu zespołowego, którego efektem ma być praca inżynierska.
Dokument zawiera opis przypadków użycia oraz wymagania funkcjonalne oraz proponowany podział prac.

\section{Opis programu}
Celem projektu jest stworzenie bazy danych szeregów czasowych efektywnie korzystającej z wydajności oferowanej przez współczesne
procesory graficzne. Dane powinny być przechowywane i przetwarzane w pamięci kart graficznych. Baza danych oprócz przechowywania
danych powinna umożliwiać szybką ich agregację oraz filtrowanie.

\section{Wymagania funkcjonalne}
    \subsection{Zapisywanie danych}
    Baza danych powinna pozwalać na przechowywanie danych (niekoniecznie w trwały sposób) napływających z różnych źródeł
    równolegle.
    \subsection{Agregowanie i filtracja danych}
    Baza danych będzie zwracała dane z zadanych okresów czasu oraz dla wyspecyfikowanych kwalifikatorów (tag, seria). Dane zwracane będą jako lista rekordów zawierających tag, serię, czas i wartość lub jako lista wartości zagregowana przy pomocy predefiniowanych funkcji.
        \subsubsection{Funkcje agregujące}
        \begin{itemize}
            \item Sumowanie
            \item Średnia
            \item Max
            \item Min
            \item Odchylenie standardowe
            \item Liczba rekordów
            \item Liczba rekordów o zadanej wartości
            \item Wariancja
            \item Całka
            \item Różniczka
            \item Histogram
        \end{itemize}
    \subsection{Działanie w czasie rzeczywistym (on-line)}
    Wszelkie operacje na danych będą wykonywane w czasie rzeczywistym.
    \subsection{Komunikacja}
    Baza danych będzie eksponowała interfejs do komunikacji ze źródłem danych oparty na protokole HTTP i formacie JSON

\section{Przypadki użycia}
    Przedstawione poniżej przypadki użycia przedstawiają podstawowy przebieg operacji.
    \begin{enumerate}
        \item Konfiguracja systemu
        \\Użytkownik chcąc zmienić podstawowe ustawienia systemu korzysta z konfiguracyjnego pliku tekstowego w którym
        zmienia żądane wartości
        \item Uruchomienie/Restart systemu
        \\Użytkownik chce uruchomić/zrestartować system korzystając z jednego polecenia na serwerze głównym
        \item Wprowadzenie danych
        \\Użytkownik chce wprowadzić nowe dane do systemu. W tym celu przygotowuje żądanie HTTP i wysyła je do
        głównego serwera
        \item Odczyt i agregacja danych
        \\Użytkownik chce odczytać dane z systemu i zastosować na nich wybraną metodę agregacji. W tym celu przygotowuje
        żądanie HTTP i wysyła je do głównego serwera. W odpowiedzi otrzymuje wynik zadanej operacji

    \end{enumerate}

\section{Harmonogram prac}
    \subsection{Metodologia}
    Projekt prowadzony będzie zgodnie z modelem przyrostowym. Na każdym punkcie kontrolnym przedstawiona będzie
    aplikacja rozszerzona o nowe funkcjonalności stanowiące dodatkową wartość dla systemu.

    \subsection{Etapy projektu}
    Wyodrębniono następujące etapy projektu
        \subsubsection{Zapisywanie danych do pamięci karty graficznej}
            System powinien być w stanie przyjąć dane z zewnątrz i zapisać je do pamięci jednej z kart graficznych na jednym z połączonych węzłów.
        \subsubsection{Ekstrakcja danych z bazy danych}
            System powinien móc odpowiedzieć na zadane przez użytkownika zapytania dotyczące przechowywanych danych
        \subsubsection{Poprawa wydajności}
            Na tym etapie cała funkcjonalność powinna być już zaimplementowana. Nastąpi finalna integracja całego systemu
            oraz wdrożenie na wydziałowy klaster obliczeniowy. Dodane zostaną testy integracyjne, zbieranie informacji o stanie węzłów oraz właściwy "load balancing". Celem tego etapu będzie eliminacja błędów oraz poprawa wydajności systemu. Poprzez wykonanie testów wydajności powinny zostać przygotowane wskazówki konfiguracji systemu zarówno dla wdrożeń na systemy typu PC jak i serwerowe.
    \subsection{Terminarz}
        \begin{enumerate}
            \item 21-25.10.2013 - Specyfikacja funkcjonalna + stworzenie środowiska pracy, repozytoriów, uzgodnienie technologii i wstępnej architektury systemu
            \item 11-15.11.2013 - Oddanie pierwszego prototypu oraz konkretnej (końcowej) specyfikacji technicznej
            \item 9-13.12.2013 - Oddanie drugiego prototypu oraz testów jednostkowych
            \item 6-10.01.2014 - Oddanie właściwego projektu, testów integracyjnych i wydajnościowych + dokumentacja końcowa
        \end{enumerate}
    \subsection{Podział pracy}
    Przewiduje się poniższy podział pracy:
        \begin{enumerate}
            \item Jakub Dutkowski
                \begin{itemize}
                    \item ...
                \end{itemize}
            \item Karol Dzitkowski
                \begin{itemize}
                    \item stworzenie struktury systemu dla węzłów rozproszonej bazy danych
                    \item obsługa wprowadzania danych (operacja Insert) dla węzłów
                    \item obsługa zapytań przychodzących z serwera głównego i zarządzanie zadaniami na węzłach
                    \item przetwarzanie/zbieranie i wstępna agregacja danych zwróconych z GPU na węzłach
                    \item przygotowywanie odpowiedzi dla zadań
                \end{itemize}
            \item Tomasz Janiszewski
                \begin{itemize}
                    \item komunikacja pomiędzy bazą danych a klientem
                    \item komunikacja sieciową pomiędzy elementami systemu
                \end{itemize}
            \item Wspólna praca
                \begin{itemize}
                    \item testowanie
                    \item integracja
                \end{itemize}
        \end{enumerate}

\end{document}